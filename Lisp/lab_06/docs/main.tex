\subsection*{Задание 1}
Написать хвостовую рекурсивную функцию my-reverse, которая развернет верхний уровень своего списка-аргумента lst.

\begin{lstinputlisting}[
	label={lst:t1},
	style={lsp},
	]{./src/task_1.lisp}
\end{lstinputlisting}

\subsection*{Задание 2}

Написать функцию, которая возвращает первый элемент списка - аргумента, который сам
является непустым списком.

\begin{lstinputlisting}[
	label={lst:t2},
	style={lsp},
	]{./src/task_2.lisp}
\end{lstinputlisting}

\subsection*{Задание 3}

Напишите рекурсивную функцию, которая умножает на заданное число-аргумент все
числа из заданного списка-аргумента, когда \\
a) все элементы списка --- числа, \\
б) элементы списка -- любые объекты.

\begin{lstinputlisting}[
	label={lst:t3},
	style={lsp},
	]{./src/task_3.lisp}
\end{lstinputlisting}

\subsection*{Задание 4}

Напишите функцию, select-between, которая из списка-аргумента, содержащего только
числа, выбирает только те, которые расположены между двумя указанными границамиаргументами и возвращает их в виде списка (упорядоченного по возрастанию списка чисел
(+ 2 балла)).

\begin{lstinputlisting}[
	label={lst:t4},
	style={lsp},
	]{./src/task_4.lisp}
\end{lstinputlisting}

\subsection*{Задание 5}

Написать рекурсивную версию (с именем rec-add) вычисления суммы чисел заданного
списка: \\
а) одноуровнего смешанного, \\
б) структурированного.

\begin{lstinputlisting}[
	label={lst:t5},
	style={lsp},
	]{./src/task_5.lisp}
\end{lstinputlisting}