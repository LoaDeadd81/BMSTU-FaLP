\chapter*{Практические задания}

\section*{Задание 1}

Чем принципиально отличаются функции cons, list, append?

cons является базовой функцией. list и append реализованы через cons.

cons является чистой функцией и принимает 2 параметра. Она создаёт списочную ячейку, в которой car указывает на первый элемент, а cdr на второй.

list является формой, так как принимает произвольное количество аргументов. Возвращает список из аргументов.

append является формой, так как принимает произвольное количество аргументов. Возвращает конкатенацию аргументов. Она возвращает точечную пару, car указывает на конкатенацию всех переданных аргументов, кроме последнего, а cdr на последний аргумент.

Пусть (setf lst1 '( a b c)) \\
(setf lst2 '( d e)). 

Каковы результаты вычисления следующих выражений?

\begin{lstinputlisting}[
	caption={Задание 1},
	label={lst:t1},
	style={lsp},
	]{./src/task_1.lisp}
\end{lstinputlisting}

\clearpage

\section*{Задание 2}

Каковы результаты вычисления следующих выражений, и почему?

\begin{lstinputlisting}[
	caption={Задание 2},
	label={lst:t2},
	style={lsp},
	]{./src/task_2.lisp}
\end{lstinputlisting}

\section*{Задание 3}

Написать, по крайней мере, два варианта функции, которая возвращает последний элемент своего списка-аргумента. 

\begin{lstinputlisting}[
	caption={Задание 3},
	label={lst:t3},
	style={lsp},
	]{./src/task_3.lisp}
\end{lstinputlisting}

\clearpage

\section*{Задание 4}

Написать, по крайней мере, два варианта функции, которая возвращает свой список аргумент без последнего элемента. 

\begin{lstinputlisting}[
	caption={Задание 4},
	label={lst:t4},
	style={lsp},
	]{./src/task_4.lisp}
\end{lstinputlisting}

\section*{Задание 5}

Напишите функцию swap-first-last, которая переставляет в списке аргументе первый и последний элементы.

\begin{lstinputlisting}[
	caption={Задание 5},
	label={lst:t5},
	style={lsp},
	]{./src/task_5.lisp}
\end{lstinputlisting}

\section*{Задание 6}

Написать простой вариант игры в кости, в котором бросаются две правильные кости. Если сумма выпавших очков равна 7 или 11 — выигрыш, если выпало (1,1) или (6,6) — игрок имеет право снова бросить кости, во всех остальных случаях ход переходит ко второму игроку, но запоминается сумма выпавших очков. Если второй игрок не выигрывает абсолютно, то выигрывает тот игрок, у которого больше очков. Результат игры и значения выпавших костей выводить на экран с помощью функции print. 

\begin{lstinputlisting}[
	caption={Задание 6},
	label={lst:t6},
	style={lsp},
	]{./src/task_6.lisp}
\end{lstinputlisting}

\clearpage

\section*{Задание 7}

Написать функцию, которая по своему списку-аргументу lst определяет является ли он палиндромом (то есть равны ли lst и (reverse lst)).

\begin{lstinputlisting}[
	caption={Задание 7},
	label={lst:t7},
	style={lsp},
	]{./src/task_7.lisp}
\end{lstinputlisting}

\section*{Задание 8}

Напишите свои необходимые функции, которые обрабатывают таблицу из 4-х точечных пар: (страна . столица), и возвращают по стране - столицу, а по столице — страну.

\begin{lstinputlisting}[
	caption={Задание 8},
	label={lst:t8},
	style={lsp},
	]{./src/task_8.lisp}
\end{lstinputlisting}

\section*{Задание 9}

Напишите функцию, которая умножает на заданное число-аргумент первый числовой элемент списка из заданного 3-х элементного списка аргумента, когда \\
a) все элементы списка --- числа, \\
6) элементы списка -- любые объекты.

\begin{lstinputlisting}[
	caption={Задание 9},
	label={lst:t9},
	style={lsp},
	]{./src/task_9.lisp}
\end{lstinputlisting}