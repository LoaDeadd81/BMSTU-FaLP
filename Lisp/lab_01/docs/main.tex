\chapter{Теоретические вопросы}

\section*{Вопрос 1. Элементы языка: определение, синтаксис, представление в памяти}
\subsection*{Элементы языка и их определение}

Вся информация (данные и программмы) в Lisp представляется в виде символных выражений	 --- \texttt{S-выражений}. По определению:

\texttt{S-выражение := <атом>|<точечная пара>}

Атомами являются: символы -- набор литер, специальные символы -- {T, Nil},самоопределимые атомы -- числа, строки.

Более сложные данные -- списки и точечные пары (структуры) строятся из унифицированных структур -- бинарных узлов.

Точечная пара -- структура данных, состоящая из двух символьных выражений, разделенных точкой. По определению:

\noindent{\texttt{Точечные пары ::= (<атом>, <атом>) | (<атом>, <точечная пара>) |}}

{\texttt{(<точечная пара>, <атом>) | (<точечная пара>, <точечная пара>)}}

Список -- это структура, которая может быть пустой и непустой. Если непустой, то состоит из двух элементов: голова -- любой элемент, хвост -- список. По определению:

\noindent{\texttt{Список ::= <пустой список> | <непустой список>)}}, где

{\texttt{<пустой список> ::= () | Nil}},

{\texttt{<непустой список> ::= (<первый элемент>, <хвост>)}},

{\texttt{<первый элемент> ::= (S-выражение)}},

{\texttt{<хвост> ::= <список>}}

\subsection*{Синтаксис элементов языка и их представление в памяти}

Элементарной синтаксической конструкцией языка является атом. Атом начинающийся с цифры или унарного знака считается числовым, иначе символьным. 

Любая структура (точечная пара или список) заключается в {\texttt{()}}. Пример:
\begin{itemize}[label=---]
	\item {\texttt{(A . B)}} --- точечная пара
	\item {\texttt{(A)}} --- список из одного элемента
\end{itemize}
Пустой список изображается как {\texttt{Nil}} или {\texttt{()}}. Непустой список может быть записан как {\texttt{(A . (B . (C . Nil)))}} или {\texttt{(A B C)}}. Элементами списка могут быть атомы или другие структуры. Любая структура представляется в памяти с помощью бинарных узлов (списковых ячеек), которые хранят два указателя: на голову и хвост.


	
\section*{Вопрос 2. Особенности языка Lisp. Структура программы. Символ апостроф}

Lisp --- интерпретируемый символьный язык программирования, т.~е. язык предназначен для символных вычислений и преобразований. В основе языка лежит идея $\lambda$-исчисления.

Программа и данные в Lisp представлены списками. Синтаксического различия между программой и данными нет.
Это позволяет использовать программу как данные и заставить её менять саму себя.
По умолчанию список считается вычислимой формой, в которой первый элемент --- название функции, остальные элементы --- аргументы функции.

Поскольку программа и данные представлены списками они неразличимы. Для их различия существует функция {\texttt{quote}}, блокирующая вычисление. Символ {\texttt{'}} является её сокращенным обозначением.

\section*{Вопрос 3. Базис языка Lisp. Ядро языка}

Базис --- это минимально необходимый набор конструкций, к которому могут быть сведены другие конструкции языка.

Базис Lisp образуют:
\begin{enumerate}[label=\arabic*)]
	\item атомы;
	\item структуры;
	\item базовые функции;
	\item функционалы.
\end{enumerate}

\chapter{Практические задания}

\section*{Задание 1}
Представить следующие списки в виде списочных ячеек:
\begin{enumerate}[label=\arabic*)]
	\item \texttt{'(open close halph)}
	\item \texttt{'((open1) (close2) (halph3))}
	\item \texttt{'((one) for all (and (me (for you))))}
	\item \texttt{'((TOOL) (call))}
	\item \texttt{'((TOOL1) ((call2)) ((sell)))}
	\item \texttt{'(((TOOL) (call)) ((sell)))}
\end{enumerate}

Решение:
\imgScale{0.5}{lab_01_1_1}{\texttt{'(open close halph)}}
\imgScale{0.5}{lab_01_1_2}{\texttt{'((open1) (close2) (halph3))}}
\imgScale{0.4}{lab_01_1_3}{\texttt{'((one) for all (and (me (for you))))}}
\imgScale{0.5}{lab_01_1_4}{\texttt{'((TOOL) (call))}}
\imgScale{0.5}{lab_01_1_5}{\texttt{'((TOOL1) ((call2)) ((sell)))}}
\clearpage
\imgScale{0.5}{lab_01_1_6}{\texttt{'(((TOOL) (call)) ((sell)))}}


\section*{Задание 2}
Используя только функции \texttt{CAR} и \texttt{CDR},  написать выражения, возвращающие:
\begin{enumerate}[label=\arabic*)]
	\item второй;
	\item третий;
	\item четвёртый;
\end{enumerate}
элементы заданного списка.

Решение: 

\begin{lstinputlisting}[
	caption={Задание 2},
	label={lst:t2},
	style={lsp},
	]{./src/task_2.lisp}
\end{lstinputlisting}


\section*{Задание 3}

Что будет в результате вычисления выражений?
\begin{lstinputlisting}[
	caption={Задание 3},
	label={lst:t3},
	style={lsp},
	]{./src/task_3.lisp}
\end{lstinputlisting}

\section*{Задание 4}
Напишите результат вычисления выражений и объясните как он получен:
\begin{lstinputlisting}[
	caption={Задание 4},
	label={lst:t4},
	style={lsp},
	]{./src/task_4.lisp}
\end{lstinputlisting}

\section*{Задание 5}

Написать $\lambda$-выражение и соответствующую функцию и представить результаты в виде списочных ячеек.

\begin{itemize}[label=---]
	\item написать функцию {\texttt{(f ar1 ar2 ar3 ar4)}}, возвращающую список \\{\texttt{((ar1 ar2) (ar3 ar4))}};
	\item написать функцию {\texttt{(f ar1 ar2)}}, возвращающую список {\texttt{((ar1) (ar2))}};
	\item {\texttt{(f ar1)}}, возвращающую список {\texttt{(((ar1)))}}.
\end{itemize}


\begin{lstinputlisting}[
	caption={Задание 5},
	label={lst:t5},
	style={lsp},
	]{./src/task_5.lisp}
\end{lstinputlisting}

Результаты в виде списочных ячеек представлены на рисунках 

\imgScale{0.5}{lab_01_5_1}{\texttt{((ar1 ar2) (ar3 ar4))}}
\imgScale{0.5}{lab_01_5_2}{\texttt{((ar1) (ar2))}}
\imgScale{0.5}{lab_01_5_3}{\texttt{(((ar1)))}}