\chapter{Практические задания}

\section*{Задание 1}
Представить следующие списки в виде списочных ячеек:
\begin{enumerate}[label=\arabic*)]
	\item \texttt{'(open close halph)}
	\item \texttt{'((open1) (close2) (halph3))}
	\item \texttt{'((one) for all (and (me (for you))))}
	\item \texttt{'((TOOL) (call))}
	\item \texttt{'((TOOL1) ((call2)) ((sell)))}
	\item \texttt{'(((TOOL) (call)) ((sell)))}
\end{enumerate}

Решение:
\imgScale{0.5}{lab_01_1_1}{\texttt{'(open close halph)}}
\imgScale{0.5}{lab_01_1_2}{\texttt{'((open1) (close2) (halph3))}}
\imgScale{0.4}{lab_01_1_3}{\texttt{'((one) for all (and (me (for you))))}}
\imgScale{0.5}{lab_01_1_4}{\texttt{'((TOOL) (call))}}
\imgScale{0.5}{lab_01_1_5}{\texttt{'((TOOL1) ((call2)) ((sell)))}}
\clearpage
\imgScale{0.5}{lab_01_1_6}{\texttt{'(((TOOL) (call)) ((sell)))}}


\section*{Задание 2}
Используя только функции \texttt{CAR} и \texttt{CDR},  написать выражения, возвращающие:
\begin{enumerate}[label=\arabic*)]
	\item второй;
	\item третий;
	\item четвёртый;
\end{enumerate}
элементы заданного списка.

Решение: 

\begin{lstinputlisting}[
	caption={Задание 2},
	label={lst:t2},
	style={lsp},
	]{./src/task_2.lisp}
\end{lstinputlisting}


\section*{Задание 3}

Что будет в результате вычисления выражений?
\clearpage
\begin{lstinputlisting}[
	caption={Задание 3},
	label={lst:t3},
	style={lsp},
	]{./src/task_3.lisp}
\end{lstinputlisting}

\section*{Задание 4}
Напишите результат вычисления выражений и объясните как он получен:
\begin{lstinputlisting}[
	caption={Задание 4},
	label={lst:t4},
	style={lsp},
	]{./src/task_4.lisp}
\end{lstinputlisting}

\section*{Задание 5}

Написать $\lambda$-выражение и соответствующую функцию и представить результаты в виде списочных ячеек.

\begin{itemize}[label=---]
	\item написать функцию {\texttt{(f ar1 ar2 ar3 ar4)}}, возвращающую список \\{\texttt{((ar1 ar2) (ar3 ar4))}};
	\item написать функцию {\texttt{(f ar1 ar2)}}, возвращающую список {\texttt{((ar1) (ar2))}};
	\item {\texttt{(f ar1)}}, возвращающую список {\texttt{(((ar1)))}}.
\end{itemize}


\begin{lstinputlisting}[
	caption={Задание 5},
	label={lst:t5},
	style={lsp},
	]{./src/task_5.lisp}
\end{lstinputlisting}

Результаты в виде списочных ячеек представлены на рисунках 

\imgScale{0.5}{lab_01_5_1}{\texttt{((ar1 ar2) (ar3 ar4))}}
\imgScale{0.5}{lab_01_5_2}{\texttt{((ar1) (ar2))}}
\imgScale{0.5}{lab_01_5_3}{\texttt{(((ar1)))}}