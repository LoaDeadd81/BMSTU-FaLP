\chapter*{Практические задания}

\section*{Задание 1}

Составить диаграмму вычисления следующих выражений: \\
(equal 3 (abs - 3)) \\
(equal (* 2 3) (+ 7 2)) \\
(equal (+ 1 2) 3) \\
(equal (- 7 3) (* 3 2)) \\
(equal (* 4 7) 21) \\
(equal (abs (- 2 4)) 3)) \\
\clearpage

\section*{Задание 2}
Написать функцию, вычисляющую гипотенузу прямоугольного треугольника по заданным катетам и составить диаграмму её вычисления.

\begin{lstinputlisting}[
	caption={Задание 2},
	label={lst:t2},
	style={lsp},
	]{./src/task_2.lisp}
\end{lstinputlisting}

\section*{Задание 3}
Каковы результаты вычисления следующих выражений?(объяснить возможную ошибку и варианты ее устранения)

\begin{lstinputlisting}[
	caption={Задание 3},
	label={lst:t3},
	style={lsp},
	]{./src/task_3.lisp}
\end{lstinputlisting}

Возможные ошибки:\\
The variable ... is unbound --- обращение к символу, которому не было присвоено значение.\\
invalid number of arguments --- в функции передано некорректное количество аргументов. \\
The value ... is not of type NUMBER --- все ожидаемого типа NUMBER какой-то другой

\section*{Задание 4}
Написать функцию longer\_then от двух списков-аргументов, которая возвращает Т, если первый аргумент имеет большую длину.

\begin{lstinputlisting}[
	caption={Задание 4},
	label={lst:t4},
	style={lsp},
	]{./src/task_4.lisp}
\end{lstinputlisting}

\section*{Задание 5}
Каковы результаты вычисления следующих выражений?

\begin{lstinputlisting}[
	caption={Задание 5},
	label={lst:t5},
	style={lsp},
	]{./src/task_5.lisp}
\end{lstinputlisting}

\section*{Задание 6}
Дана функция (defun mystery (x) (list (second x) (first x))). Какие результаты вычисления следующих выражений? 

\begin{lstinputlisting}[
	caption={Задание 6},
	label={lst:t6},
	style={lsp},
	]{./src/task_6.lisp}
\end{lstinputlisting}

\section*{Задание 7}
Написать функцию, которая переводит температуру в системе Фаренгейта температуру по Цельсию (defum f-to-c (temp)…).

\begin{lstinputlisting}[
	caption={Задание 7},
	label={lst:t7},
	style={lsp},
	]{./src/task_7.lisp}
\end{lstinputlisting}

\section*{Задание 8}
Что получится при вычисления каждого из выражений?

\begin{lstinputlisting}[
	caption={Задание 8},
	label={lst:t8},
	style={lsp},
	]{./src/task_8.lisp}
\end{lstinputlisting}