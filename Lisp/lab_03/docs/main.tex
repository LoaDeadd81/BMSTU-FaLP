\chapter*{Практические задания}

\section*{Задание 1}

Написать функцию, которая принимает целое число и возвращает первое четное число, не меньшее аргумента.

\begin{lstinputlisting}[
	caption={Задание 1},
	label={lst:t1},
	style={lsp},
	]{./src/task_1.lisp}
\end{lstinputlisting}

\section*{Задание 2}
Написать функцию, которая принимает число и возвращает число того же знака, но с модулем на 1 больше модуля аргумента.

\begin{lstinputlisting}[
	caption={Задание 2},
	label={lst:t2},
	style={lsp},
	]{./src/task_2.lisp}
\end{lstinputlisting}

\section*{Задание 3}
Написать функцию, которая принимает два числа и возвращает список из этих чисел, расположенный по возрастанию.

\begin{lstinputlisting}[
	caption={Задание 3},
	label={lst:t3},
	style={lsp},
	]{./src/task_3.lisp}
\end{lstinputlisting}

\section*{Задание 4}
Написать функцию, которая принимает три числа и возвращает Т только тогда, когда первое число расположено между вторым и третьим.

\begin{lstinputlisting}[
	caption={Задание 4},
	label={lst:t4},
	style={lsp},
	]{./src/task_4.lisp}
\end{lstinputlisting}

\section*{Задание 5}
Каковы результаты вычисления следующих выражений?

\begin{lstinputlisting}[
	caption={Задание 5},
	label={lst:t5},
	style={lsp},
	]{./src/task_5.lisp}
\end{lstinputlisting}

\section*{Задание 6}
Написать предикат, который принимает два числа-аргумента и возвращает Т, если первое число не меньше второго.

\begin{lstinputlisting}[
	caption={Задание 6},
	label={lst:t6},
	style={lsp},
	]{./src/task_6.lisp}
\end{lstinputlisting}

\section*{Задание 7}
Какой из следующих двух вариантов предиката ошибочен и почему?

\begin{enumerate}
	\item (defun pred1 (x) (and (numberp x) (plusp x)))
	\item (defun pred2 (x) (and (plusp x) (numberp x)))
\end{enumerate}

Второй предикат является ошибочным, так как в нём сначала проверяется является ли x положительным, а только потом является ли x числом. Это может вызвать ошибку в предикате plusp, если x не число. В первом же предикате, если x не число, то numberp вернёт Nil и plusp вычисляться не будет. 

\section*{Задание 8}
Решить задачу 4, используя для ее решения конструкции: только IF, только COND, только AND/OR.

\begin{lstinputlisting}[
	caption={Задание 8},
	label={lst:t8},
	style={lsp},
	]{./src/task_8.lisp}
\end{lstinputlisting}

\section*{Задание 9}
Решить задачу 4, используя для ее решения конструкции: только IF, только COND, только AND/OR.

\begin{lstinputlisting}[
	caption={Задание 9},
	label={lst:t9},
	style={lsp},
	]{./src/task_9.lisp}
\end{lstinputlisting}