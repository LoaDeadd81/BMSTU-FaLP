\subsection*{Задание}
Создать базу знаний <<Собственник>>, дополнив (и минимально изменив) базу знаний, хранящую знания
\begin{itemize}
	\item <<Телефонный справочник>>: Фамилия, №тел, Адрес – структура (Город, Улица, №дома, №кв),
	\item <<Автомобили>>: Фамилия\_владельца, Марка, Цвет, Стоимость, и др.,
	\item <<Вкладчики банков>>: Фамилия, Банк, счет, сумма, др.,
\end{itemize}
знаниями о дополнительной собственности владельца. Преобразовать знания об
автомобиле к форме знаний о собственности.
Вид собственности (кроме автомобиля):
\begin{itemize}
	\item Строение, стоимость и другие его характеристики
	\item Участок, стоимость и другие его характеристики;
	\item Водный\_транспорт, стоимость и другие его характеристики.
\end{itemize}

Описать и использовать вариантный домен: Собственность. Владелец может иметь, но
только один объект каждого вида собственности (это касается и автомобиля), или не
иметь некоторых видов собственности.
Используя конъюнктивное правило и разные формы задания одного вопроса (пояснять
для какого задания – какой вопрос),
обеспечить возможность поиска:
\begin{itemize}
	\item Названий всех объектов собственности заданного субъекта,
	\item Названий и стоимости всех объектов собственности заданного субъекта,
	\item * Разработать правило, позволяющее найти суммарную стоимость всех объектов собственности заданного субъекта.
\end{itemize}


\begin{lstinputlisting}[
	label={lst:t1},
	]{./src/task_1.pro}
\end{lstinputlisting}

Для 2-го пункт и одной фамилии составить таблицу, отражающую конкретный
порядок работы системы, с объяснениями порядка работы и особенностей использования
доменов (указать конкретные Т1 и Т2 и полную подстановку на каждом шаге)

Вопрос:  property\_and\_price\_by\_surname(''Ivanov'', PropertyName, Price).