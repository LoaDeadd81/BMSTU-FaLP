%\chapter{Лис}
\section*{Задание 1}
Создать базу знаний  «Предки», позволяющую наиболее эффективным способом (за   меньшее   количество   шагов,   что   обеспечивается   меньшим   количеством предложений БЗ - правил), и используя разные варианты (примеры) простого вопроса, (указать: какой вопрос для какого варианта) определить:
\begin{enumerate}
	\item По имени субъекта определить всех его бабушек (предки 2-го колена).
	\item По имени субъекта определить всех его дедушек (предки 2-го колена).
	\item По имени субъекта определить всех его бабушек и дедушек (предки 2-го
колена).
	\item По имени субъекта определить его бабушку по материнской линии (предки 2-го
колена).
	\item По имени субъекта определить его бабушку и дедушку по материнской линии
(предки 2-го колена).
\end{enumerate}

Минимизировать количество правил и количество вариантов вопросов. Использовать
конъюнктивные правила и простой вопрос. Для одного из вариантов ВОПРОСА задания 1
составить таблицу, отражающую конкретный порядок работы системы.

\begin{lstinputlisting}[
	label={lst:t1},
	]{../src/main.pro}
\end{lstinputlisting}

\clearpage

\section*{Задание 2}

Дополнить базу знаний правилами, позволяющими найти:
\begin{enumerate}
	\item Максимум из двух чисел без использования отсечения.
	\item Максимум из двух чисел с использованием отсечения.
	\item Максимум из трёх чисел без использования отсечения.
	\item Максимум из трёх чисел с использованием отсечения.
\end{enumerate}

Убедиться в правильности результатов.
Для каждого случая пункта 2 обосновать необходимость всех условий тела.
Для одного из вариантов ВОПРОСА и каждого варианта задания 2 составить
таблицу, отражающую конкретный порядок работы системы.

Для одного из вариантов ВОПРОСА и конкретной БЗ составить таблицу, отражающую
конкретный порядок работы системы, с объяснениями: очередная проблема на каждом шаге
и метод ее решения; каково новое текущее состояние резольвенты, как получено; какие
дальнейшие   действия?   (Запускается   ли   алгоритм   унификации?   Каких   термов?   Почему
этих?) ; вывод по результатам очередного шага и дальнейшие действия.

\begin{lstinputlisting}[
	label={lst:t1},
	]{../src/task2.pro}
\end{lstinputlisting}

